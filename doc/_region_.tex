\message{ !name(rendu1.tex)}% Created 2014-12-02 Tue 16:56
\documentclass[11pt]{article}
\usepackage[utf8]{inputenc}
\usepackage[T1]{fontenc}
\usepackage{fixltx2e}
\usepackage{graphicx}
\usepackage{longtable}
\usepackage{array}
\usepackage{float}
\usepackage{wrapfig}
\usepackage{rotating}
\usepackage[normalem]{ulem}
\usepackage{amsmath}
\usepackage{textcomp}
\usepackage{marvosym}
\usepackage{wasysym}
\usepackage{amssymb}
\usepackage{hyperref}
\usepackage[a4paper, margin=1cm]{geometry}
\author{Corentin Cadiou, Aymeric Fromherz, Lucas Verney}
\date{\today}
\title{Système digital, rapport de projet}
\begin{document}

\message{ !name(rendu1.tex) !offset(5) }
\section{Architecture}
Les microcontrolleurs ATTiny4 sont des microcontrolleurs 8bits avec un
jeu de 54 instructions pour 16 registres 8bits, qui fonctionnent
généralement à une dizaine de $MHz$. Ils possèdent en outre 512 octets
de ROM et 32 octets de RAM. En plus des 16 registres généraux, les
ATTiny4 possèdent d'autres registres utiles à implémenter pour :
\begin{itemize}
  \item les \emph{input/output}
\end{itemize}
\subsection{Registres}

\subsection{Mémoire}

\message{ !name(rendu1.tex) !offset(151) }

\end{document}
